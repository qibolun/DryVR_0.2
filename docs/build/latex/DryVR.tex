%% Generated by Sphinx.
\def\sphinxdocclass{report}
\documentclass[letterpaper,10pt,english]{sphinxmanual}
\ifdefined\pdfpxdimen
   \let\sphinxpxdimen\pdfpxdimen\else\newdimen\sphinxpxdimen
\fi \sphinxpxdimen=.75bp\relax

\usepackage[utf8]{inputenc}
\ifdefined\DeclareUnicodeCharacter
 \ifdefined\DeclareUnicodeCharacterAsOptional\else
  \DeclareUnicodeCharacter{00A0}{\nobreakspace}
\fi\fi
\usepackage{cmap}
\usepackage[T1]{fontenc}
\usepackage{amsmath,amssymb,amstext}
\usepackage{babel}
\usepackage{times}
\usepackage[Bjarne]{fncychap}
\usepackage{longtable}
\usepackage{sphinx}

\usepackage{geometry}
\usepackage{multirow}
\usepackage{eqparbox}

% Include hyperref last.
\usepackage{hyperref}
% Fix anchor placement for figures with captions.
\usepackage{hypcap}% it must be loaded after hyperref.
% Set up styles of URL: it should be placed after hyperref.
\urlstyle{same}

\addto\captionsenglish{\renewcommand{\figurename}{Fig.}}
\addto\captionsenglish{\renewcommand{\tablename}{Table}}
\addto\captionsenglish{\renewcommand{\literalblockname}{Listing}}

\addto\extrasenglish{\def\pageautorefname{page}}

\setcounter{tocdepth}{1}



\title{DryVR Documentation}
\date{Jan 31, 2018}
\release{2.0}
\author{Chuchu Fan, Bolun Qi}
\newcommand{\sphinxlogo}{}
\renewcommand{\releasename}{Release}
\makeindex

\begin{document}

\maketitle
\sphinxtableofcontents
\phantomsection\label{\detokenize{index::doc}}

\begin{quote}\begin{description}
\item[{Release}] \leavevmode
2.0

\item[{Date}] \leavevmode
01/30/2018

\end{description}\end{quote}

DryVR is a framework for verifying cyber-physical systems. It specifically handles systems that are described by a combination of a {\hyperref[\detokenize{dryvr's_language:black-box-label}]{\sphinxcrossref{\DUrole{std,std-ref}{Black-box Simulator}}}} for trajectories and a white-box {\hyperref[\detokenize{dryvr's_language:transition-graph-label}]{\sphinxcrossref{\DUrole{std,std-ref}{Transition Graph}}}} specifying mode switches. The framework uses a probabilistic algorithm for learning sensitivity of the continuous trajectories from simulation data and includes a bounded reachability analysis algorithm that uses the learned sensitivity.


\chapter{Status}
\label{\detokenize{status:status}}\label{\detokenize{status::doc}}\label{\detokenize{status:welcome-to-dryvr-s-user-manual}}
Jan 24.2018. DryVR 2.0 is done. Adding state dependent transition and control synthesis.

April 18.2017. The installation is tested on Ubuntu 16.04 (64 bit version).

March 23.2017. The tool is tested on Ubuntu 16.04 (64 bit version).


\chapter{Installation}
\label{\detokenize{installtion:installation}}\label{\detokenize{installtion::doc}}
To install the required packages, please run:

\begin{sphinxVerbatim}[commandchars=\\\{\}]
\PYG{n}{sudo} \PYG{o}{.}\PYG{o}{/}\PYG{n}{installRequirement}\PYG{o}{.}\PYG{n}{sh}
\end{sphinxVerbatim}

The current version of installation file has been tested on a clean
install of Ubuntu 16.04. If you wish to install DryVR on other versions of Linux operation system, please make sure the following packages are correctly installed.

To install packages indepently, the following will be required:
\begin{itemize}
\item {} 
python 2.7

\item {} 
numpy

\item {} 
scipy

\item {} 
sympy

\item {} 
matplotlib

\item {} 
python igraph

\item {} 
python Z3

\item {} 
glpk(4.39 or ealier eversion)

\item {} 
pyglpk

\item {} 
python-cairo

\item {} 
python tk

\item {} 
gmpc

\item {} 
graphviz

\item {} 
pygraphviz

\end{itemize}


\chapter{Usage}
\label{\detokenize{usage:usage}}\label{\detokenize{usage::doc}}

\section{Run DryVR Verfication}
\label{\detokenize{usage:run-dryvr-verfication}}
To run DryVR verfication, please run:

\begin{sphinxVerbatim}[commandchars=\\\{\}]
\PYG{n}{python} \PYG{n}{main}\PYG{o}{.}\PYG{n}{py} \PYG{n+nb}{input}\PYG{o}{/}\PYG{o}{*}\PYG{o}{/}\PYG{p}{[}\PYG{n}{input\PYGZus{}file}\PYG{p}{]}
\end{sphinxVerbatim}

for example:

\begin{sphinxVerbatim}[commandchars=\\\{\}]
\PYG{n}{python} \PYG{n}{main}\PYG{o}{.}\PYG{n}{py} \PYG{n+nb}{input}\PYG{o}{/}\PYG{n}{daginput}\PYG{o}{/}\PYG{n}{input\PYGZus{}thermo}\PYG{o}{.}\PYG{n}{json}
\end{sphinxVerbatim}


\section{Run DryVR Control Synthesis}
\label{\detokenize{usage:run-dryvr-control-synthesis}}
To run DryVR graph search algorithm, please run:

\begin{sphinxVerbatim}[commandchars=\\\{\}]
\PYG{n}{python} \PYG{n}{rrt}\PYG{o}{.}\PYG{n}{py} \PYG{n+nb}{input}\PYG{o}{/}\PYG{o}{*}\PYG{o}{/}\PYG{p}{[}\PYG{n}{input\PYGZus{}file}\PYG{p}{]}
\end{sphinxVerbatim}

for example:

\begin{sphinxVerbatim}[commandchars=\\\{\}]
\PYG{n}{python} \PYG{n}{rrt}\PYG{o}{.}\PYG{n}{py} \PYG{n+nb}{input}\PYG{o}{/}\PYG{n}{rrtinput}\PYG{o}{/}\PYG{n}{mazefinder}\PYG{o}{.}\PYG{n}{json}
\end{sphinxVerbatim}


\section{Plotter}
\label{\detokenize{usage:plotter}}
After you run the our tool, a reachtube.txt file will be generated in output folder unless the model is determined unsafe during simulation test.

To plot the reachtube, please run:

\begin{sphinxVerbatim}[commandchars=\\\{\}]
\PYG{n}{python} \PYG{n}{plotter}\PYG{o}{.}\PYG{n}{py} \PYG{o}{\PYGZhy{}}\PYG{n}{x} \PYG{p}{[}\PYG{n}{x} \PYG{n}{dimension} \PYG{n}{number}\PYG{p}{]} \PYG{o}{\PYGZhy{}}\PYG{n}{y} \PYG{p}{[}\PYG{n}{y} \PYG{n}{dimension} \PYG{n}{number} \PYG{n+nb}{list}\PYG{p}{]} \PYG{o}{\PYGZhy{}}\PYG{n}{f} \PYG{p}{[}\PYG{n+nb}{input} \PYG{n}{file} \PYG{n}{name}\PYG{p}{]} \PYG{o}{\PYGZhy{}}\PYG{n}{o} \PYG{p}{[}\PYG{n}{output} \PYG{n}{file} \PYG{n}{name}\PYG{p}{]}
\end{sphinxVerbatim}

-x is the dimension number for x-axis, the default value will be 0, which is the dimension of time.

-y is dimension number lists indicates the dimension you want to draw for y-axis. For example -y {[}1,2{]}. The default value will be {[}1{]}.

-f is the file path for reach tube file that you want to plot, the default value will be output/reachtube.txt.

-o is output file option, the default value is plotResult.png.

To get help for plotter, please run:

\begin{sphinxVerbatim}[commandchars=\\\{\}]
\PYG{n}{python} \PYG{n}{plotter}\PYG{o}{.}\PYG{n}{py} \PYG{o}{\PYGZhy{}}\PYG{n}{h}
\end{sphinxVerbatim}

Note that the dimension 0 is local time and last dimension is global time. For example, input\_AEB's inital set is {[}{[}0.0,-23.0,0.0,1.0,0.0,-15.0,0.0,1.0{]},{[}0.0,-22.8,0.0,1.0,0.0,-15.0,0.0,1.0{]}{]}. Therefore, it has 8 dimensions in total. You can choose to plot dimension from 0 to 9. Where dimension 0 is the local time and dimension 9 is global time. Dimension 1\textasciitilde{}8 is corresponding to the dimension you specify in initial set.

for example:

\begin{sphinxVerbatim}[commandchars=\\\{\}]
\PYG{n}{python} \PYG{n}{plotter}\PYG{o}{.}\PYG{n}{py} \PYG{o}{\PYGZhy{}}\PYG{n}{y} \PYG{p}{[}\PYG{l+m+mi}{1}\PYG{p}{,}\PYG{l+m+mi}{2}\PYG{p}{]} \PYG{o}{\PYGZhy{}}\PYG{n}{f} \PYG{n}{output}\PYG{o}{/}\PYG{n}{reachtube}\PYG{o}{.}\PYG{n}{txt}
\end{sphinxVerbatim}

More plot results can be found at the {\hyperref[\detokenize{example:example-label}]{\sphinxcrossref{\DUrole{std,std-ref}{Examples}}}} page.


\chapter{DryVR's Language}
\label{\detokenize{dryvr's_language:dryvr-s-language}}\label{\detokenize{dryvr's_language::doc}}
In DryVR,  a hybrid system is modeled as a combination of a white-box that specifies the mode switches ({\hyperref[\detokenize{dryvr's_language:transition-graph-label}]{\sphinxcrossref{\DUrole{std,std-ref}{Transition Graph}}}}) and a black-box that can simulate the continuous evolution in each mode ({\hyperref[\detokenize{dryvr's_language:black-box-label}]{\sphinxcrossref{\DUrole{std,std-ref}{Black-box Simulator}}}}).


\section{Black-box Simulator}
\label{\detokenize{dryvr's_language:black-box-simulator}}\label{\detokenize{dryvr's_language:black-box-label}}
The black-box simulator for a (deterministic) takes as input a mode label, an initial state \(x_0\), and a finite
sequence of time points \(t_1, \ldots, t_k\), and returns a sequence of
states \(sim(mode,x_0,t_1), \ldots, sim(mode,x_0,t_k)\)
as the simulation trajectory of the system in the given mode starting from \(x_0\) at the time points \(t_1, \ldots, t_k\).

DryVR uses the black-box simulator by calling the simulation function:

\begin{sphinxVerbatim}[commandchars=\\\{\}]
\PYG{n}{TC\PYGZus{}Simulate}\PYG{p}{(}\PYG{n}{Modes}\PYG{p}{,}\PYG{n}{initialCondition}\PYG{p}{,}\PYG{n}{time\PYGZus{}bound}\PYG{p}{)}
\end{sphinxVerbatim}

Given the mode name \sphinxquotedblleft{}Mode\sphinxquotedblright{}, initial state \sphinxquotedblleft{}initialCondition\sphinxquotedblright{}  and time horizon \sphinxquotedblleft{}time\_bound\sphinxquotedblright{}, the function TC\_Simulate should return an python array of the form:

\begin{sphinxVerbatim}[commandchars=\\\{\}]
\PYG{p}{[}\PYG{p}{[}\PYG{n}{t\PYGZus{}0}\PYG{p}{,}\PYG{n}{variable\PYGZus{}1}\PYG{p}{(}\PYG{n}{t\PYGZus{}0}\PYG{p}{)}\PYG{p}{,}\PYG{n}{variable\PYGZus{}2}\PYG{p}{(}\PYG{n}{t\PYGZus{}0}\PYG{p}{)}\PYG{p}{,}\PYG{o}{.}\PYG{o}{.}\PYG{o}{.}\PYG{p}{]}\PYG{p}{,}\PYG{p}{[}\PYG{n}{t\PYGZus{}1}\PYG{p}{,}\PYG{n}{variable\PYGZus{}1}\PYG{p}{(}\PYG{n}{t\PYGZus{}1}\PYG{p}{)}\PYG{p}{,}\PYG{n}{variable\PYGZus{}2}\PYG{p}{(}\PYG{n}{t\PYGZus{}1}\PYG{p}{)}\PYG{p}{,}\PYG{o}{.}\PYG{o}{.}\PYG{o}{.}\PYG{p}{]}\PYG{p}{,}\PYG{o}{.}\PYG{o}{.}\PYG{o}{.}\PYG{p}{]}
\end{sphinxVerbatim}

We provide several example simulation functions and you have to write your own if you want to verify systems that use other black-boxes. Once you create the TC\_Simulate function and corresponding input file, you can run DryVR to check the safety of your system. To connect DryVR with your own black-box simulator, please refer to section {\hyperref[\detokenize{dryvr's_language:advance-label}]{\sphinxcrossref{\DUrole{std,std-ref}{Advanced Tricks: Verify your own black-box system}}}} for more details.


\section{Transition Graph}
\label{\detokenize{dryvr's_language:transition-graph-label}}\label{\detokenize{dryvr's_language:transition-graph}}\begin{wrapfigure}{r}{0pt}
\centering
\noindent\sphinxincludegraphics[scale=0.6]{{curgraph}.png}
\caption{The transition of Automatic Emergency Braking (AEB) system}\label{\detokenize{dryvr's_language:id1}}\end{wrapfigure}

A transition graph is a labeled, directed acyclic graph as shown on the right. The vertex labels (red nodes in the graph) specify the modes of the system, and the edge labels specify the transition time from the predecessor node to the successor node.

The transition graph shown on the right defines an automatic emergency braking system. Car1 is driving ahead of Car2 on a straight lane. Initially, both car1 and car2 are in the constant speed mode (Const;Const). Within a short amount of time ({[}0,0.1{]}s) Car1 transits into brake mode while Car2 remains in the cruise mode (Brk;Const). After {[}0.8,0.9{]}s, Car2 will react by braking as well so both cars are in the brake mode (Brk;Brk).

The transition graph will be generated automatically by DryVR and stored in the tool's root directory as curgraph.png


\section{Input Format}
\label{\detokenize{dryvr's_language:input-format}}\label{\detokenize{dryvr's_language:input-format-label}}
The input for DryVR is of the form

\begin{sphinxVerbatim}[commandchars=\\\{\}]
\PYG{p}{\PYGZob{}}
  \PYG{l+s+s2}{\PYGZdq{}}\PYG{l+s+s2}{vertex}\PYG{l+s+s2}{\PYGZdq{}}\PYG{p}{:}\PYG{p}{[}\PYG{n}{transition} \PYG{n}{graph} \PYG{n}{vertex} \PYG{n}{labels} \PYG{p}{(}\PYG{n}{modes}\PYG{p}{)}\PYG{p}{]}
  \PYG{l+s+s2}{\PYGZdq{}}\PYG{l+s+s2}{edge}\PYG{l+s+s2}{\PYGZdq{}}\PYG{p}{:}\PYG{p}{[}\PYG{n}{transition} \PYG{n}{graph} \PYG{n}{edges}\PYG{p}{,} \PYG{p}{(}\PYG{n}{i}\PYG{p}{,}\PYG{n}{j}\PYG{p}{)} \PYG{n}{means} \PYG{n}{there} \PYG{o+ow}{is} \PYG{n}{a} \PYG{n}{directed} \PYG{n}{edge} \PYG{k+kn}{from} \PYG{n+nn}{vertex} \PYG{n}{i} \PYG{n}{to} \PYG{n}{vertex} \PYG{n}{j}\PYG{p}{]}
  \PYG{l+s+s2}{\PYGZdq{}}\PYG{l+s+s2}{variables}\PYG{l+s+s2}{\PYGZdq{}}\PYG{p}{:}\PYG{p}{[}\PYG{n}{the} \PYG{n}{name} \PYG{n}{of} \PYG{n}{variables} \PYG{o+ow}{in} \PYG{n}{the} \PYG{n}{system}\PYG{p}{]}
  \PYG{l+s+s2}{\PYGZdq{}}\PYG{l+s+s2}{guards}\PYG{l+s+s2}{\PYGZdq{}}\PYG{p}{:}\PYG{p}{[}\PYG{n}{transition} \PYG{n}{graph} \PYG{n}{edge} \PYG{n}{labels} \PYG{p}{(}\PYG{n}{transition} \PYG{n}{condition}\PYG{p}{)}\PYG{p}{]}
  \PYG{l+s+s2}{\PYGZdq{}}\PYG{l+s+s2}{resets}\PYG{l+s+s2}{\PYGZdq{}}\PYG{p}{:}\PYG{p}{[}\PYG{n}{reset} \PYG{n}{condition} \PYG{n}{after} \PYG{n}{transition}\PYG{p}{]} \PYG{c+c1}{\PYGZsh{} This is optional if you do not want reset}
  \PYG{l+s+s2}{\PYGZdq{}}\PYG{l+s+s2}{initialMode}\PYG{l+s+s2}{\PYGZdq{}}\PYG{p}{:}\PYG{p}{[}\PYG{n}{label} \PYG{k}{for} \PYG{n}{initial} \PYG{n}{mode}\PYG{p}{]} \PYG{c+c1}{\PYGZsh{} This is optional for DAG graph}
  \PYG{l+s+s2}{\PYGZdq{}}\PYG{l+s+s2}{initialSet}\PYG{l+s+s2}{\PYGZdq{}}\PYG{p}{:}\PYG{p}{[}\PYG{n}{two} \PYG{n}{arrays} \PYG{n}{defining} \PYG{n}{the} \PYG{n}{lower} \PYG{o+ow}{and} \PYG{n}{upper} \PYG{n}{bound} \PYG{n}{of} \PYG{n}{each} \PYG{n}{variable}\PYG{p}{]}
  \PYG{l+s+s2}{\PYGZdq{}}\PYG{l+s+s2}{unsafeSet}\PYG{l+s+s2}{\PYGZdq{}}\PYG{p}{:}\PYG{o}{@}\PYG{p}{[}\PYG{n}{mode} \PYG{n}{name}\PYG{p}{]}\PYG{p}{:}\PYG{p}{[}\PYG{n}{unsafe} \PYG{n}{region}\PYG{p}{]}
  \PYG{l+s+s2}{\PYGZdq{}}\PYG{l+s+s2}{timeHorizon}\PYG{l+s+s2}{\PYGZdq{}}\PYG{p}{:}\PYG{p}{[}\PYG{n}{Time} \PYG{n}{bound} \PYG{k}{for} \PYG{n}{the} \PYG{n}{verification}\PYG{p}{]}
  \PYG{l+s+s2}{\PYGZdq{}}\PYG{l+s+s2}{directory}\PYG{l+s+s2}{\PYGZdq{}}\PYG{p}{:} \PYG{n}{directory} \PYG{n}{of} \PYG{n}{the} \PYG{n}{folder} \PYG{n}{which} \PYG{n}{contains} \PYG{n}{the} \PYG{n}{simulator} \PYG{k}{for} \PYG{n}{black}\PYG{o}{\PYGZhy{}}\PYG{n}{box} \PYG{n}{system}
  \PYG{l+s+s2}{\PYGZdq{}}\PYG{l+s+s2}{bloatingMethod}\PYG{l+s+s2}{\PYGZdq{}}\PYG{p}{:} \PYG{n}{specify} \PYG{n}{the} \PYG{n}{bloating} \PYG{n}{method}\PYG{p}{,} \PYG{n}{which} \PYG{n}{can} \PYG{n}{be} \PYG{n}{either} \PYG{l+s+s2}{\PYGZdq{}}\PYG{l+s+s2}{PW}\PYG{l+s+s2}{\PYGZdq{}} \PYG{o+ow}{or} \PYG{l+s+s2}{\PYGZdq{}}\PYG{l+s+s2}{GLOBAL}\PYG{l+s+s2}{\PYGZdq{}} \PYG{c+c1}{\PYGZsh{} This is optional, if you don\PYGZsq{}t have this field in input file, DryVR will use GLOBAL as default bloating method.}
  \PYG{l+s+s2}{\PYGZdq{}}\PYG{l+s+s2}{kvalue}\PYG{l+s+s2}{\PYGZdq{}}\PYG{p}{:} \PYG{n}{specify} \PYG{n}{the} \PYG{n}{k}\PYG{o}{\PYGZhy{}}\PYG{n}{value} \PYG{n}{that} \PYG{n}{used} \PYG{n}{by} \PYG{n}{piecewise} \PYG{n}{bloating} \PYG{n}{method} \PYG{c+c1}{\PYGZsh{} This field must be specified if you choose the bloatingMethod to \PYGZdq{}PW\PYGZdq{}}
\PYG{p}{\PYGZcb{}}
\end{sphinxVerbatim}

Some fields are optional in DryVR's input langauge such as resets, initialMode, bloatingMethod and kvalue under some conditions. Please read the comment.

Example input for the Automatic Emergency Braking System

\begin{sphinxVerbatim}[commandchars=\\\{\}]
\PYG{p}{\PYGZob{}}
  \PYG{l+s+s2}{\PYGZdq{}}\PYG{l+s+s2}{vertex}\PYG{l+s+s2}{\PYGZdq{}}\PYG{p}{:}\PYG{p}{[}\PYG{l+s+s2}{\PYGZdq{}}\PYG{l+s+s2}{Const;Const}\PYG{l+s+s2}{\PYGZdq{}}\PYG{p}{,}\PYG{l+s+s2}{\PYGZdq{}}\PYG{l+s+s2}{Brk;Const}\PYG{l+s+s2}{\PYGZdq{}}\PYG{p}{,}\PYG{l+s+s2}{\PYGZdq{}}\PYG{l+s+s2}{Brk;Brk}\PYG{l+s+s2}{\PYGZdq{}}\PYG{p}{]}\PYG{p}{,}
  \PYG{l+s+s2}{\PYGZdq{}}\PYG{l+s+s2}{edge}\PYG{l+s+s2}{\PYGZdq{}}\PYG{p}{:}\PYG{p}{[}\PYG{p}{[}\PYG{l+m+mi}{0}\PYG{p}{,}\PYG{l+m+mi}{1}\PYG{p}{]}\PYG{p}{,}\PYG{p}{[}\PYG{l+m+mi}{1}\PYG{p}{,}\PYG{l+m+mi}{2}\PYG{p}{]}\PYG{p}{]}\PYG{p}{,}
  \PYG{l+s+s2}{\PYGZdq{}}\PYG{l+s+s2}{variables}\PYG{l+s+s2}{\PYGZdq{}}\PYG{p}{:}\PYG{p}{[}\PYG{l+s+s2}{\PYGZdq{}}\PYG{l+s+s2}{car1\PYGZus{}x}\PYG{l+s+s2}{\PYGZdq{}}\PYG{p}{,}\PYG{l+s+s2}{\PYGZdq{}}\PYG{l+s+s2}{car1\PYGZus{}y}\PYG{l+s+s2}{\PYGZdq{}}\PYG{p}{,}\PYG{l+s+s2}{\PYGZdq{}}\PYG{l+s+s2}{car1\PYGZus{}vx}\PYG{l+s+s2}{\PYGZdq{}}\PYG{p}{,}\PYG{l+s+s2}{\PYGZdq{}}\PYG{l+s+s2}{car1\PYGZus{}vy}\PYG{l+s+s2}{\PYGZdq{}}\PYG{p}{,}\PYG{l+s+s2}{\PYGZdq{}}\PYG{l+s+s2}{car2\PYGZus{}x}\PYG{l+s+s2}{\PYGZdq{}}\PYG{p}{,}\PYG{l+s+s2}{\PYGZdq{}}\PYG{l+s+s2}{car2\PYGZus{}y}\PYG{l+s+s2}{\PYGZdq{}}\PYG{p}{,}\PYG{l+s+s2}{\PYGZdq{}}\PYG{l+s+s2}{car2\PYGZus{}vx}\PYG{l+s+s2}{\PYGZdq{}}\PYG{p}{,}\PYG{l+s+s2}{\PYGZdq{}}\PYG{l+s+s2}{car2\PYGZus{}vy}\PYG{l+s+s2}{\PYGZdq{}}\PYG{p}{]}\PYG{p}{,}
  \PYG{l+s+s2}{\PYGZdq{}}\PYG{l+s+s2}{guards}\PYG{l+s+s2}{\PYGZdq{}}\PYG{p}{:}\PYG{p}{[}
    \PYG{l+s+s2}{\PYGZdq{}}\PYG{l+s+s2}{And(t\PYGZgt{}0.0,t\PYGZlt{}=0.1)}\PYG{l+s+s2}{\PYGZdq{}}\PYG{p}{,}
    \PYG{l+s+s2}{\PYGZdq{}}\PYG{l+s+s2}{And(t\PYGZgt{}0.8,t\PYGZlt{}=0.9)}\PYG{l+s+s2}{\PYGZdq{}}
  \PYG{p}{]}\PYG{p}{,}
  \PYG{l+s+s2}{\PYGZdq{}}\PYG{l+s+s2}{initialSet}\PYG{l+s+s2}{\PYGZdq{}}\PYG{p}{:}\PYG{p}{[}\PYG{p}{[}\PYG{l+m+mf}{0.0}\PYG{p}{,}\PYG{l+m+mf}{0.5}\PYG{p}{,}\PYG{l+m+mf}{0.0}\PYG{p}{,}\PYG{l+m+mf}{1.0}\PYG{p}{,}\PYG{l+m+mf}{0.0}\PYG{p}{,}\PYG{o}{\PYGZhy{}}\PYG{l+m+mf}{17.0}\PYG{p}{,}\PYG{l+m+mf}{0.0}\PYG{p}{,}\PYG{l+m+mf}{1.0}\PYG{p}{]}\PYG{p}{,}\PYG{p}{[}\PYG{l+m+mf}{0.0}\PYG{p}{,}\PYG{l+m+mf}{1.0}\PYG{p}{,}\PYG{l+m+mf}{0.0}\PYG{p}{,}\PYG{l+m+mf}{1.0}\PYG{p}{,}\PYG{l+m+mf}{0.0}\PYG{p}{,}\PYG{o}{\PYGZhy{}}\PYG{l+m+mf}{15.0}\PYG{p}{,}\PYG{l+m+mf}{0.0}\PYG{p}{,}\PYG{l+m+mf}{1.0}\PYG{p}{]}\PYG{p}{]}\PYG{p}{,}
  \PYG{l+s+s2}{\PYGZdq{}}\PYG{l+s+s2}{unsafeSet}\PYG{l+s+s2}{\PYGZdq{}}\PYG{p}{:}\PYG{l+s+s2}{\PYGZdq{}}\PYG{l+s+s2}{@Allmode:And(car1\PYGZus{}y\PYGZhy{}car2\PYGZus{}y\PYGZlt{}3, car2\PYGZus{}y\PYGZhy{}car1\PYGZus{}y\PYGZlt{}3)}\PYG{l+s+s2}{\PYGZdq{}}\PYG{p}{,}
  \PYG{l+s+s2}{\PYGZdq{}}\PYG{l+s+s2}{timeHorizon}\PYG{l+s+s2}{\PYGZdq{}}\PYG{p}{:}\PYG{l+m+mf}{5.0}\PYG{p}{,}
  \PYG{l+s+s2}{\PYGZdq{}}\PYG{l+s+s2}{directory}\PYG{l+s+s2}{\PYGZdq{}}\PYG{p}{:}\PYG{l+s+s2}{\PYGZdq{}}\PYG{l+s+s2}{examples/cars}\PYG{l+s+s2}{\PYGZdq{}}
\PYG{p}{\PYGZcb{}}
\end{sphinxVerbatim}


\section{Output Interpretation}
\label{\detokenize{dryvr's_language:output-interpretation}}
The tool will print background information like the current mode, transition time, initial set on the run. The final result about goal reached/cannot find graph will be printed at the bottom.

When the system find transition graph, the final result will look like

\begin{sphinxVerbatim}[commandchars=\\\{\}]
System is Safe!
\end{sphinxVerbatim}

When the system is unsafe from simulation, the final result will look like

\begin{sphinxVerbatim}[commandchars=\\\{\}]
\PYG{n}{Current} \PYG{n}{simulation} \PYG{o+ow}{is} \PYG{o+ow}{not} \PYG{n}{safe}\PYG{o}{.} \PYG{n}{Program} \PYG{n}{halt}
\end{sphinxVerbatim}

When the system is unsafe from verification, the final result will look like

\begin{sphinxVerbatim}[commandchars=\\\{\}]
\PYG{n}{System} \PYG{o+ow}{is} \PYG{o+ow}{not} \PYG{n}{safe} \PYG{o+ow}{in} \PYG{n}{Mode} \PYG{p}{[}\PYG{n}{Mode} \PYG{n}{name}\PYG{p}{]}
\end{sphinxVerbatim}

When the system is unknown from verification, the final result will look like

\begin{sphinxVerbatim}[commandchars=\\\{\}]
\PYG{n}{Hit} \PYG{n}{refine} \PYG{n}{threshold}\PYG{p}{,} \PYG{n}{system} \PYG{n}{halt}\PYG{p}{,} \PYG{n}{result} \PYG{n}{unknown}
\end{sphinxVerbatim}

If the simulation result is not safe, the unsafe simulation trajectory will be stored in \sphinxquotedblleft{}output/Traj.txt\sphinxquotedblright{}.
Otherwise the last simulation result will be stored in \sphinxquotedblleft{}Traj.txt\sphinxquotedblright{}.

If the verfication result is not safe, the counter example reachtube will be stored in \sphinxquotedblleft{}output/unsafeTube.txt\sphinxquotedblright{}.


\section{Advanced Tricks: Verify your own black-box system}
\label{\detokenize{dryvr's_language:advance-label}}\label{\detokenize{dryvr's_language:advanced-tricks-verify-your-own-black-box-system}}
We use a very simple example of a thermostat as the starting point to show how to use DryVR to verify your own black-box system.

The thermostat is a one-dimensional linear hybrid system with two modes \sphinxquotedblleft{}On\sphinxquotedblright{} and \sphinxquotedblleft{}Off\sphinxquotedblright{}. The only state variable is the temperature \(x\). In the \sphinxquotedblleft{}On\sphinxquotedblright{} mode, the system dynamic is
\begin{equation*}
\begin{split}\dot{x} = 0.1 x,\end{split}
\end{equation*}
and in the \sphinxquotedblleft{}Off\sphinxquotedblright{} mode, the system dynamic is
\begin{equation*}
\begin{split}\dot{x} = -0.1 x,\end{split}
\end{equation*}
As for DryVR, of course, all the information about dynamics is hidden. Instead, you need to provide the simulator function TC\_Simulate as discussed in {\hyperref[\detokenize{dryvr's_language:black-box-label}]{\sphinxcrossref{\DUrole{std,std-ref}{Black-box Simulator}}}}.

\sphinxstylestrong{Step 1}:
Create a folder in the DryVR root directory for your new model and enter it.

\begin{sphinxVerbatim}[commandchars=\\\{\}]
\PYG{n}{cd} \PYG{n}{examples}
\PYG{n}{mkdir} \PYG{n}{Thermostats}
\PYG{n}{cd} \PYG{n}{Thermostats}
\end{sphinxVerbatim}

\sphinxstylestrong{Step 2}:
Inside your model folder, create a python script for your model.

\begin{sphinxVerbatim}[commandchars=\\\{\}]
\PYG{n}{touch} \PYG{n}{Thermostats\PYGZus{}ODE}\PYG{o}{.}\PYG{n}{py}
\end{sphinxVerbatim}

\sphinxstylestrong{Step 3}: Write the TC\_Simulate function in the python file Thermostats\_ODE.py.

For the thermostat system, one simulator function could be:

\begin{sphinxVerbatim}[commandchars=\\\{\}]
\PYG{k}{def} \PYG{n+nf}{thermo\PYGZus{}dynamic}\PYG{p}{(}\PYG{n}{y}\PYG{p}{,}\PYG{n}{t}\PYG{p}{,}\PYG{n}{rate}\PYG{p}{)}\PYG{p}{:}
    \PYG{n}{dydt} \PYG{o}{=} \PYG{n}{rate}\PYG{o}{*}\PYG{n}{y}
    \PYG{k}{return} \PYG{n}{dydt}

\PYG{k}{def} \PYG{n+nf}{TC\PYGZus{}Simulate}\PYG{p}{(}\PYG{n}{Mode}\PYG{p}{,}\PYG{n}{initialCondition}\PYG{p}{,}\PYG{n}{time\PYGZus{}bound}\PYG{p}{)}\PYG{p}{:}
    \PYG{n}{time\PYGZus{}step} \PYG{o}{=} \PYG{l+m+mf}{0.05}\PYG{p}{;}
    \PYG{n}{time\PYGZus{}bound} \PYG{o}{=} \PYG{n+nb}{float}\PYG{p}{(}\PYG{n}{time\PYGZus{}bound}\PYG{p}{)}
    \PYG{n}{initial} \PYG{o}{=} \PYG{p}{[}\PYG{n+nb}{float}\PYG{p}{(}\PYG{n}{tmp}\PYG{p}{)}  \PYG{k}{for} \PYG{n}{tmp} \PYG{o+ow}{in} \PYG{n}{initialCondition}\PYG{p}{]}
    \PYG{n}{number\PYGZus{}points} \PYG{o}{=} \PYG{n+nb}{int}\PYG{p}{(}\PYG{n}{np}\PYG{o}{.}\PYG{n}{ceil}\PYG{p}{(}\PYG{n}{time\PYGZus{}bound}\PYG{o}{/}\PYG{n}{time\PYGZus{}step}\PYG{p}{)}\PYG{p}{)}
    \PYG{n}{t} \PYG{o}{=} \PYG{p}{[}\PYG{n}{i}\PYG{o}{*}\PYG{n}{time\PYGZus{}step} \PYG{k}{for} \PYG{n}{i} \PYG{o+ow}{in} \PYG{n+nb}{range}\PYG{p}{(}\PYG{l+m+mi}{0}\PYG{p}{,}\PYG{n}{number\PYGZus{}points}\PYG{p}{)}\PYG{p}{]}
    \PYG{k}{if} \PYG{n}{t}\PYG{p}{[}\PYG{o}{\PYGZhy{}}\PYG{l+m+mi}{1}\PYG{p}{]} \PYG{o}{!=} \PYG{n}{time\PYGZus{}step}\PYG{p}{:}
        \PYG{n}{t}\PYG{o}{.}\PYG{n}{append}\PYG{p}{(}\PYG{n}{time\PYGZus{}bound}\PYG{p}{)}

    \PYG{n}{y\PYGZus{}initial} \PYG{o}{=} \PYG{n}{initial}\PYG{p}{[}\PYG{l+m+mi}{0}\PYG{p}{]}

    \PYG{k}{if} \PYG{n}{Mode} \PYG{o}{==} \PYG{l+s+s1}{\PYGZsq{}}\PYG{l+s+s1}{On}\PYG{l+s+s1}{\PYGZsq{}}\PYG{p}{:}
        \PYG{n}{rate} \PYG{o}{=} \PYG{l+m+mf}{0.1}
    \PYG{k}{elif} \PYG{n}{Mode} \PYG{o}{==} \PYG{l+s+s1}{\PYGZsq{}}\PYG{l+s+s1}{Off}\PYG{l+s+s1}{\PYGZsq{}}\PYG{p}{:}
        \PYG{n}{rate} \PYG{o}{=} \PYG{o}{\PYGZhy{}}\PYG{l+m+mf}{0.1}
    \PYG{k}{else}\PYG{p}{:}
        \PYG{n+nb}{print}\PYG{p}{(}\PYG{l+s+s1}{\PYGZsq{}}\PYG{l+s+s1}{Wrong Mode name!}\PYG{l+s+s1}{\PYGZsq{}}\PYG{p}{)}
    \PYG{n}{sol} \PYG{o}{=} \PYG{n}{odeint}\PYG{p}{(}\PYG{n}{thermo\PYGZus{}dynamic}\PYG{p}{,}\PYG{n}{y\PYGZus{}initial}\PYG{p}{,}\PYG{n}{t}\PYG{p}{,}\PYG{n}{args}\PYG{o}{=}\PYG{p}{(}\PYG{n}{rate}\PYG{p}{,}\PYG{p}{)}\PYG{p}{,}\PYG{n}{hmax} \PYG{o}{=} \PYG{n}{time\PYGZus{}step}\PYG{p}{)}

    \PYG{c+c1}{\PYGZsh{} Construct the final output}
    \PYG{n}{trace} \PYG{o}{=} \PYG{p}{[}\PYG{p}{]}
    \PYG{k}{for} \PYG{n}{j} \PYG{o+ow}{in} \PYG{n+nb}{range}\PYG{p}{(}\PYG{n+nb}{len}\PYG{p}{(}\PYG{n}{t}\PYG{p}{)}\PYG{p}{)}\PYG{p}{:}
        \PYG{n}{tmp} \PYG{o}{=} \PYG{p}{[}\PYG{p}{]}
        \PYG{n}{tmp}\PYG{o}{.}\PYG{n}{append}\PYG{p}{(}\PYG{n}{t}\PYG{p}{[}\PYG{n}{j}\PYG{p}{]}\PYG{p}{)}
        \PYG{n}{tmp}\PYG{o}{.}\PYG{n}{append}\PYG{p}{(}\PYG{n}{sol}\PYG{p}{[}\PYG{n}{j}\PYG{p}{,}\PYG{l+m+mi}{0}\PYG{p}{]}\PYG{p}{)}
        \PYG{n}{trace}\PYG{o}{.}\PYG{n}{append}\PYG{p}{(}\PYG{n}{tmp}\PYG{p}{)}
    \PYG{k}{return} \PYG{n}{trace}
\end{sphinxVerbatim}

In this example, we use odeint simulator from Scipy, but you use any programming language as long as the TC\_Simulate function follows the input-output requirement:

\begin{sphinxVerbatim}[commandchars=\\\{\}]
\PYG{n}{TC\PYGZus{}Simulate}\PYG{p}{(}\PYG{n}{Mode}\PYG{p}{,}\PYG{n}{initialCondition}\PYG{p}{,}\PYG{n}{time\PYGZus{}bound}\PYG{p}{)}
\PYG{n}{Input}\PYG{p}{:}
    \PYG{n}{Mode} \PYG{p}{(}\PYG{n}{string}\PYG{p}{)} \PYG{o}{\PYGZhy{}}\PYG{o}{\PYGZhy{}} \PYG{n}{a} \PYG{n}{string} \PYG{n}{indicates} \PYG{n}{the} \PYG{n}{model} \PYG{n}{you} \PYG{n}{want} \PYG{n}{to} \PYG{n}{simulate}\PYG{o}{.} \PYG{n}{Ex}\PYG{o}{.} \PYG{l+s+s2}{\PYGZdq{}}\PYG{l+s+s2}{On}\PYG{l+s+s2}{\PYGZdq{}}
    \PYG{n}{initialCondition} \PYG{p}{(}\PYG{n+nb}{list} \PYG{n}{of} \PYG{n+nb}{float}\PYG{p}{)} \PYG{o}{\PYGZhy{}}\PYG{o}{\PYGZhy{}} \PYG{n}{a} \PYG{n+nb}{list} \PYG{n}{contains} \PYG{n}{the} \PYG{n}{initial} \PYG{n}{condition}\PYG{o}{.} \PYG{n}{Ex}\PYG{o}{.} \PYG{l+s+s2}{\PYGZdq{}}\PYG{l+s+s2}{[32.0]}\PYG{l+s+s2}{\PYGZdq{}}
    \PYG{n}{time\PYGZus{}bound} \PYG{p}{(}\PYG{n+nb}{float}\PYG{p}{)} \PYG{o}{\PYGZhy{}}\PYG{o}{\PYGZhy{}} \PYG{n}{a} \PYG{n+nb}{float} \PYG{n}{indicates} \PYG{n}{the} \PYG{n}{time} \PYG{n}{horizon} \PYG{k}{for} \PYG{n}{simulation}\PYG{o}{.} \PYG{n}{EX}\PYG{o}{.} \PYG{l+s+s1}{\PYGZsq{}}\PYG{l+s+s1}{10.0}\PYG{l+s+s1}{\PYGZsq{}}
\PYG{n}{Output}\PYG{p}{:}
    \PYG{n}{Trace} \PYG{p}{(}\PYG{n+nb}{list} \PYG{n}{of} \PYG{n+nb}{list} \PYG{n}{of} \PYG{n+nb}{float}\PYG{p}{)} \PYG{o}{\PYGZhy{}}\PYG{o}{\PYGZhy{}} \PYG{n}{a} \PYG{n+nb}{list} \PYG{n}{of} \PYG{n}{lists} \PYG{n}{contain} \PYG{n}{the} \PYG{n}{trace} \PYG{k+kn}{from} \PYG{n+nn}{a} \PYG{n}{simulation}\PYG{o}{.}
    \PYG{n}{Each} \PYG{n}{index} \PYG{n}{represents} \PYG{n}{the} \PYG{n}{simulation} \PYG{k}{for} \PYG{n}{certain} \PYG{n}{time} \PYG{n}{step}\PYG{o}{.}\PYG{n}{Represents} \PYG{k}{as} \PYG{p}{[}\PYG{n}{time}\PYG{p}{,} \PYG{n}{v1}\PYG{p}{,} \PYG{n}{v2}\PYG{p}{,} \PYG{o}{.}\PYG{o}{.}\PYG{o}{.}\PYG{o}{.}\PYG{o}{.}\PYG{o}{.}\PYG{o}{.}\PYG{o}{.}\PYG{p}{]}\PYG{o}{.}
    \PYG{n}{Ex}\PYG{o}{.} \PYG{l+s+s2}{\PYGZdq{}}\PYG{l+s+s2}{[[0.0,32.0],[0.1,32.1],[0.2,32.2]........[10.0,34.3]]}\PYG{l+s+s2}{\PYGZdq{}}
\end{sphinxVerbatim}

\sphinxstylestrong{Step 4}:
Inside your model folder, create a Python initiate script.

\begin{sphinxVerbatim}[commandchars=\\\{\}]
\PYG{n}{touch} \PYG{n+nf+fm}{\PYGZus{}\PYGZus{}init\PYGZus{}\PYGZus{}}\PYG{o}{.}\PYG{n}{py}
\end{sphinxVerbatim}

Inside your initiate script, import file with function TC\_Simulate.

\begin{sphinxVerbatim}[commandchars=\\\{\}]
\PYG{k+kn}{from} \PYG{n+nn}{Thermostats\PYGZus{}ODE} \PYG{k}{import} \PYG{o}{*}
\end{sphinxVerbatim}

\sphinxstylestrong{Step 5}:
Go to inputFile folder and create an input file for your new model using the format discussed in {\hyperref[\detokenize{dryvr's_language:input-format-label}]{\sphinxcrossref{\DUrole{std,std-ref}{Input Format}}}}.

Create a transition graph specifying the mode transitions. For example, we want the temperature to start within the range \([75,76]\) in the \sphinxquotedblleft{}On\sphinxquotedblright{} mode. After \([1,1.1]\) second, it transits to the \sphinxquotedblleft{}Off\sphinxquotedblright{} mode, and transits back to the \sphinxquotedblleft{}On\sphinxquotedblright{} mode after another \([1,1.1]\) seconds. For bounded time \(3.5s\), we want to check whether the temperature is above \(90\).

The input file can be written as:

\begin{sphinxVerbatim}[commandchars=\\\{\}]
\PYG{p}{\PYGZob{}}
  \PYG{l+s+s2}{\PYGZdq{}}\PYG{l+s+s2}{vertex}\PYG{l+s+s2}{\PYGZdq{}}\PYG{p}{:}\PYG{p}{[}\PYG{l+s+s2}{\PYGZdq{}}\PYG{l+s+s2}{On}\PYG{l+s+s2}{\PYGZdq{}}\PYG{p}{,}\PYG{l+s+s2}{\PYGZdq{}}\PYG{l+s+s2}{Off}\PYG{l+s+s2}{\PYGZdq{}}\PYG{p}{,}\PYG{l+s+s2}{\PYGZdq{}}\PYG{l+s+s2}{On}\PYG{l+s+s2}{\PYGZdq{}}\PYG{p}{]}\PYG{p}{,}
  \PYG{l+s+s2}{\PYGZdq{}}\PYG{l+s+s2}{edge}\PYG{l+s+s2}{\PYGZdq{}}\PYG{p}{:}\PYG{p}{[}\PYG{p}{[}\PYG{l+m+mi}{0}\PYG{p}{,}\PYG{l+m+mi}{1}\PYG{p}{]}\PYG{p}{,}\PYG{p}{[}\PYG{l+m+mi}{1}\PYG{p}{,}\PYG{l+m+mi}{2}\PYG{p}{]}\PYG{p}{]}\PYG{p}{,}
  \PYG{l+s+s2}{\PYGZdq{}}\PYG{l+s+s2}{variables}\PYG{l+s+s2}{\PYGZdq{}}\PYG{p}{:}\PYG{p}{[}\PYG{l+s+s2}{\PYGZdq{}}\PYG{l+s+s2}{temp}\PYG{l+s+s2}{\PYGZdq{}}\PYG{p}{]}\PYG{p}{,}
  \PYG{l+s+s2}{\PYGZdq{}}\PYG{l+s+s2}{guards}\PYG{l+s+s2}{\PYGZdq{}}\PYG{p}{:}\PYG{p}{[}\PYG{l+s+s2}{\PYGZdq{}}\PYG{l+s+s2}{And(t\PYGZgt{}1.0,t\PYGZlt{}=1.1)}\PYG{l+s+s2}{\PYGZdq{}}\PYG{p}{,}\PYG{l+s+s2}{\PYGZdq{}}\PYG{l+s+s2}{And(t\PYGZgt{}1.0,t\PYGZlt{}=1.1)}\PYG{l+s+s2}{\PYGZdq{}}\PYG{p}{]}\PYG{p}{,}
  \PYG{l+s+s2}{\PYGZdq{}}\PYG{l+s+s2}{initialSet}\PYG{l+s+s2}{\PYGZdq{}}\PYG{p}{:}\PYG{p}{[}\PYG{p}{[}\PYG{l+m+mf}{75.0}\PYG{p}{]}\PYG{p}{,}\PYG{p}{[}\PYG{l+m+mf}{76.0}\PYG{p}{]}\PYG{p}{]}\PYG{p}{,}
  \PYG{l+s+s2}{\PYGZdq{}}\PYG{l+s+s2}{unsafeSet}\PYG{l+s+s2}{\PYGZdq{}}\PYG{p}{:}\PYG{l+s+s2}{\PYGZdq{}}\PYG{l+s+s2}{@Allmode:temp\PYGZgt{}91}\PYG{l+s+s2}{\PYGZdq{}}\PYG{p}{,}
  \PYG{l+s+s2}{\PYGZdq{}}\PYG{l+s+s2}{timeHorizon}\PYG{l+s+s2}{\PYGZdq{}}\PYG{p}{:}\PYG{l+m+mf}{3.5}\PYG{p}{,}
  \PYG{l+s+s2}{\PYGZdq{}}\PYG{l+s+s2}{directory}\PYG{l+s+s2}{\PYGZdq{}}\PYG{p}{:}\PYG{l+s+s2}{\PYGZdq{}}\PYG{l+s+s2}{examples/Thermostats}\PYG{l+s+s2}{\PYGZdq{}}
\PYG{p}{\PYGZcb{}}
\end{sphinxVerbatim}

Save the input file in the folder input/daginput and name it as input\_thermo.json.

\sphinxstylestrong{Step6}:
Run the verification algorithm using the command:

\begin{sphinxVerbatim}[commandchars=\\\{\}]
\PYG{n}{python} \PYG{n}{main}\PYG{o}{.}\PYG{n}{py} \PYG{n+nb}{input}\PYG{o}{/}\PYG{n}{daginput}\PYG{o}{/}\PYG{n}{input\PYGZus{}thermo}\PYG{o}{.}\PYG{n}{json}
\end{sphinxVerbatim}

The system has been checked to be safe with the output:

\begin{sphinxVerbatim}[commandchars=\\\{\}]
System is Safe!
\end{sphinxVerbatim}

We can plot the reachtube using the command:

\begin{sphinxVerbatim}[commandchars=\\\{\}]
\PYG{n}{python} \PYG{n}{plotter}\PYG{o}{.}\PYG{n}{py}
\end{sphinxVerbatim}

And the reachtube for the temperature is shown as
\begin{figure}[htbp]
\centering
\capstart

\noindent\sphinxincludegraphics[scale=0.6]{{thermostat}.png}
\caption{The reachtube for the temperature of the thermostat system example}\label{\detokenize{dryvr's_language:id2}}\end{figure}


\chapter{DryVR's Control Synthesis}
\label{\detokenize{dryvr's_control_synthesis:dryvr-s-control-synthesis}}\label{\detokenize{dryvr's_control_synthesis::doc}}
In DryVR,  a hybrid system is modeled as a combination of a white-box that specifies the mode switches ({\hyperref[\detokenize{dryvr's_language:transition-graph-label}]{\sphinxcrossref{\DUrole{std,std-ref}{Transition Graph}}}}) and a black-box that can simulate the continuous evolution in each mode ({\hyperref[\detokenize{dryvr's_language:black-box-label}]{\sphinxcrossref{\DUrole{std,std-ref}{Black-box Simulator}}}}).

The control synthesis problem for DryVR is to find a white-box transition graph given the black-box simulator with addition inputs listed in ({\hyperref[\detokenize{dryvr's_control_synthesis:input-format-control-label}]{\sphinxcrossref{\DUrole{std,std-ref}{Input Format}}}}).


\section{Input Format}
\label{\detokenize{dryvr's_control_synthesis:input-format}}\label{\detokenize{dryvr's_control_synthesis:input-format-control-label}}
The input for DryVR control synthesis is of the form

\begin{sphinxVerbatim}[commandchars=\\\{\}]
\PYG{p}{\PYGZob{}}
  \PYG{l+s+s2}{\PYGZdq{}}\PYG{l+s+s2}{modes}\PYG{l+s+s2}{\PYGZdq{}}\PYG{p}{:}\PYG{p}{[}\PYG{n}{modes} \PYG{n}{that} \PYG{n}{black} \PYG{n}{simulator} \PYG{n}{takes}\PYG{p}{]}
  \PYG{l+s+s2}{\PYGZdq{}}\PYG{l+s+s2}{initialMode}\PYG{l+s+s2}{\PYGZdq{}}\PYG{p}{:}\PYG{p}{[}\PYG{n}{initial} \PYG{n}{mode} \PYG{n}{that} \PYG{n}{DryVR} \PYG{n}{start} \PYG{n}{to} \PYG{n}{search}\PYG{p}{]}
  \PYG{l+s+s2}{\PYGZdq{}}\PYG{l+s+s2}{variables}\PYG{l+s+s2}{\PYGZdq{}}\PYG{p}{:}\PYG{p}{[}\PYG{n}{the} \PYG{n}{name} \PYG{n}{of} \PYG{n}{variables} \PYG{o+ow}{in} \PYG{n}{the} \PYG{n}{system}\PYG{p}{]}
  \PYG{l+s+s2}{\PYGZdq{}}\PYG{l+s+s2}{initialSet}\PYG{l+s+s2}{\PYGZdq{}}\PYG{p}{:}\PYG{p}{[}\PYG{n}{two} \PYG{n}{arrays} \PYG{n}{defining} \PYG{n}{the} \PYG{n}{lower} \PYG{o+ow}{and} \PYG{n}{upper} \PYG{n}{bound} \PYG{n}{of} \PYG{n}{each} \PYG{n}{variable}\PYG{p}{]}
  \PYG{l+s+s2}{\PYGZdq{}}\PYG{l+s+s2}{unsafeSet}\PYG{l+s+s2}{\PYGZdq{}}\PYG{p}{:}\PYG{o}{@}\PYG{p}{[}\PYG{n}{mode} \PYG{n}{name}\PYG{p}{]}\PYG{p}{:}\PYG{p}{[}\PYG{n}{unsafe} \PYG{n}{region}\PYG{p}{]}
  \PYG{l+s+s2}{\PYGZdq{}}\PYG{l+s+s2}{goalSet}\PYG{l+s+s2}{\PYGZdq{}}\PYG{p}{:}\PYG{p}{[}\PYG{n}{two} \PYG{n}{arrays} \PYG{n}{defining} \PYG{n}{the} \PYG{n}{lower} \PYG{o+ow}{and} \PYG{n}{upper} \PYG{n}{bound} \PYG{n}{of} \PYG{n}{each} \PYG{n}{variable} \PYG{k}{for} \PYG{n}{goal}\PYG{p}{]}
  \PYG{l+s+s2}{\PYGZdq{}}\PYG{l+s+s2}{timeHorizon}\PYG{l+s+s2}{\PYGZdq{}}\PYG{p}{:}\PYG{p}{[}\PYG{n}{time} \PYG{n}{bound} \PYG{k}{for} \PYG{n}{control} \PYG{n}{synthesis}\PYG{p}{,} \PYG{n}{the} \PYG{n}{graph} \PYG{n}{should} \PYG{n}{be} \PYG{n}{bounded} \PYG{o+ow}{in} \PYG{n}{time} \PYG{n}{horizon}\PYG{p}{]}
  \PYG{l+s+s2}{\PYGZdq{}}\PYG{l+s+s2}{directory}\PYG{l+s+s2}{\PYGZdq{}}\PYG{p}{:} \PYG{n}{directory} \PYG{n}{of} \PYG{n}{the} \PYG{n}{folder} \PYG{n}{which} \PYG{n}{contains} \PYG{n}{the} \PYG{n}{simulator} \PYG{k}{for} \PYG{n}{black}\PYG{o}{\PYGZhy{}}\PYG{n}{box} \PYG{n}{system}
  \PYG{l+s+s2}{\PYGZdq{}}\PYG{l+s+s2}{minTimeThres}\PYG{l+s+s2}{\PYGZdq{}}\PYG{p}{:} \PYG{n}{minimal} \PYG{n}{staying} \PYG{n}{time} \PYG{k}{for} \PYG{n}{each} \PYG{n}{mode} \PYG{n}{to} \PYG{n}{limit} \PYG{n}{number} \PYG{n}{of} \PYG{n}{trainsition}\PYG{o}{.}
  \PYG{l+s+s2}{\PYGZdq{}}\PYG{l+s+s2}{goal}\PYG{l+s+s2}{\PYGZdq{}}\PYG{p}{:}\PYG{p}{[}\PYG{p}{[}\PYG{n}{goal} \PYG{n}{variables}\PYG{p}{]}\PYG{p}{,}\PYG{p}{[}\PYG{n}{lower} \PYG{n}{bound}\PYG{p}{]}\PYG{p}{[}\PYG{n}{upper} \PYG{n}{bound}\PYG{p}{]}\PYG{p}{]} \PYG{c+c1}{\PYGZsh{} This is a rewrite for goal set for dryvr to calculate distance.}
\PYG{p}{\PYGZcb{}}
\end{sphinxVerbatim}

Example input for the robot in maze example

\begin{sphinxVerbatim}[commandchars=\\\{\}]
\PYG{p}{\PYGZob{}}
  \PYG{l+s+s2}{\PYGZdq{}}\PYG{l+s+s2}{modes}\PYG{l+s+s2}{\PYGZdq{}}\PYG{p}{:}\PYG{p}{[}\PYG{l+s+s2}{\PYGZdq{}}\PYG{l+s+s2}{0}\PYG{l+s+s2}{\PYGZdq{}}\PYG{p}{,} \PYG{l+s+s2}{\PYGZdq{}}\PYG{l+s+s2}{1}\PYG{l+s+s2}{\PYGZdq{}}\PYG{p}{,} \PYG{l+s+s2}{\PYGZdq{}}\PYG{l+s+s2}{2}\PYG{l+s+s2}{\PYGZdq{}}\PYG{p}{,} \PYG{l+s+s2}{\PYGZdq{}}\PYG{l+s+s2}{3}\PYG{l+s+s2}{\PYGZdq{}}\PYG{p}{,} \PYG{l+s+s2}{\PYGZdq{}}\PYG{l+s+s2}{4}\PYG{l+s+s2}{\PYGZdq{}}\PYG{p}{,} \PYG{l+s+s2}{\PYGZdq{}}\PYG{l+s+s2}{5}\PYG{l+s+s2}{\PYGZdq{}}\PYG{p}{,} \PYG{l+s+s2}{\PYGZdq{}}\PYG{l+s+s2}{6}\PYG{l+s+s2}{\PYGZdq{}}\PYG{p}{,} \PYG{l+s+s2}{\PYGZdq{}}\PYG{l+s+s2}{7}\PYG{l+s+s2}{\PYGZdq{}}\PYG{p}{]}\PYG{p}{,}
  \PYG{l+s+s2}{\PYGZdq{}}\PYG{l+s+s2}{initialMode}\PYG{l+s+s2}{\PYGZdq{}}\PYG{p}{:}\PYG{l+s+s2}{\PYGZdq{}}\PYG{l+s+s2}{1}\PYG{l+s+s2}{\PYGZdq{}}\PYG{p}{,}
  \PYG{l+s+s2}{\PYGZdq{}}\PYG{l+s+s2}{variables}\PYG{l+s+s2}{\PYGZdq{}}\PYG{p}{:}\PYG{p}{[}\PYG{l+s+s2}{\PYGZdq{}}\PYG{l+s+s2}{x}\PYG{l+s+s2}{\PYGZdq{}}\PYG{p}{,}\PYG{l+s+s2}{\PYGZdq{}}\PYG{l+s+s2}{y}\PYG{l+s+s2}{\PYGZdq{}}\PYG{p}{,}\PYG{l+s+s2}{\PYGZdq{}}\PYG{l+s+s2}{vx}\PYG{l+s+s2}{\PYGZdq{}}\PYG{p}{,}\PYG{l+s+s2}{\PYGZdq{}}\PYG{l+s+s2}{vy}\PYG{l+s+s2}{\PYGZdq{}}\PYG{p}{]}\PYG{p}{,}
  \PYG{l+s+s2}{\PYGZdq{}}\PYG{l+s+s2}{initialSet}\PYG{l+s+s2}{\PYGZdq{}}\PYG{p}{:}\PYG{p}{[}\PYG{p}{[}\PYG{l+m+mf}{1.0}\PYG{p}{,}\PYG{l+m+mf}{1.0}\PYG{p}{,}\PYG{l+m+mf}{1.0}\PYG{p}{,}\PYG{l+m+mf}{1.0}\PYG{p}{]}\PYG{p}{,}\PYG{p}{[}\PYG{l+m+mf}{1.1}\PYG{p}{,}\PYG{l+m+mf}{1.0}\PYG{p}{,}\PYG{l+m+mf}{1.0}\PYG{p}{,}\PYG{l+m+mf}{1.0}\PYG{p}{]}\PYG{p}{]}\PYG{p}{,}
  \PYG{l+s+s2}{\PYGZdq{}}\PYG{l+s+s2}{unsafeSet}\PYG{l+s+s2}{\PYGZdq{}}\PYG{p}{:}\PYG{l+s+s2}{\PYGZdq{}}\PYG{l+s+s2}{@Allmode:Or(And(x\PYGZgt{}=2.0, x\PYGZlt{}3.0, y\PYGZgt{}=3.0, y\PYGZlt{}=4.0), And(x\PYGZgt{}=3.0, x\PYGZlt{}=4.0, y\PYGZgt{}=2.0, y\PYGZlt{}3.0), x\PYGZlt{}0, x\PYGZgt{}5, y\PYGZlt{}0, y\PYGZgt{}5)}\PYG{l+s+s2}{\PYGZdq{}}\PYG{p}{,}
  \PYG{l+s+s2}{\PYGZdq{}}\PYG{l+s+s2}{goalSet}\PYG{l+s+s2}{\PYGZdq{}}\PYG{p}{:}\PYG{l+s+s2}{\PYGZdq{}}\PYG{l+s+s2}{And(x\PYGZgt{}=3.0, x\PYGZlt{}=4.0, y\PYGZgt{}=3.0, y\PYGZlt{}=4.0)}\PYG{l+s+s2}{\PYGZdq{}}\PYG{p}{,}
  \PYG{l+s+s2}{\PYGZdq{}}\PYG{l+s+s2}{timeHorizon}\PYG{l+s+s2}{\PYGZdq{}}\PYG{p}{:}\PYG{l+m+mf}{10.0}\PYG{p}{,}
  \PYG{l+s+s2}{\PYGZdq{}}\PYG{l+s+s2}{minTimeThres}\PYG{l+s+s2}{\PYGZdq{}}\PYG{p}{:}\PYG{l+m+mf}{1.0}\PYG{p}{,}
  \PYG{l+s+s2}{\PYGZdq{}}\PYG{l+s+s2}{directory}\PYG{l+s+s2}{\PYGZdq{}}\PYG{p}{:}\PYG{l+s+s2}{\PYGZdq{}}\PYG{l+s+s2}{examples/carinmaze}\PYG{l+s+s2}{\PYGZdq{}}\PYG{p}{,}
  \PYG{l+s+s2}{\PYGZdq{}}\PYG{l+s+s2}{goal}\PYG{l+s+s2}{\PYGZdq{}}\PYG{p}{:}\PYG{p}{[}\PYG{p}{[}\PYG{l+s+s2}{\PYGZdq{}}\PYG{l+s+s2}{x}\PYG{l+s+s2}{\PYGZdq{}}\PYG{p}{,}\PYG{l+s+s2}{\PYGZdq{}}\PYG{l+s+s2}{y}\PYG{l+s+s2}{\PYGZdq{}}\PYG{p}{]}\PYG{p}{,}\PYG{p}{[}\PYG{l+m+mf}{3.0}\PYG{p}{,}\PYG{l+m+mf}{3.0}\PYG{p}{]}\PYG{p}{,}\PYG{p}{[}\PYG{l+m+mf}{4.0}\PYG{p}{,}\PYG{l+m+mf}{4.0}\PYG{p}{]}\PYG{p}{]}
\PYG{p}{\PYGZcb{}}
\end{sphinxVerbatim}


\section{Output Interpretation}
\label{\detokenize{dryvr's_control_synthesis:output-interpretation}}
The tool will print background information like the current mode, transition time, initial set on the run. The final result about goal reached or not reached will be printed at the bottom.

When the system find the transition graph that statisfy the requirement, the final result will look like

\begin{sphinxVerbatim}[commandchars=\\\{\}]
\PYG{n}{goal} \PYG{n}{reached}
\end{sphinxVerbatim}

When the system cannot find graph, the final result will look like

\begin{sphinxVerbatim}[commandchars=\\\{\}]
\PYG{n}{could} \PYG{o+ow}{not} \PYG{n}{find} \PYG{n}{graph}
\end{sphinxVerbatim}

Note that DryVR's algorithm is searching the graph randomly, if the system cannot find the graph, it does not mean the graph is not exist with current input. You can try run the algorithm multiple times to get more accurate result.
If the the system find the transition graph, the system will plot the transition graph and will be stored in \sphinxquotedblleft{}output/rrtGraph.png\sphinxquotedblright{}


\section{Advanced Tricks: Making control synthesis work on your own black-box system}
\label{\detokenize{dryvr's_control_synthesis:advanced-tricks-making-control-synthesis-work-on-your-own-black-box-system}}
Creating black box simulator is exactly same as we introduced in DryVR's language page ({\hyperref[\detokenize{dryvr's_language:advance-label}]{\sphinxcrossref{\DUrole{std,std-ref}{Advanced Tricks: Verify your own black-box system}}}}) up to Step 4.

For the Step 5, instead of creating a verification input file, you need to create control synthesis input file we have discussed in {\hyperref[\detokenize{dryvr's_control_synthesis:input-format-control-label}]{\sphinxcrossref{\DUrole{std,std-ref}{Input Format}}}}.

For example, Let's set the intial temperature within the range \([75,76]\), and we want to reach the target temperature within the range \([68,72]\), while avoiding temperature that is larger than \(90\). We want to start our search from \sphinxquotedblleft{}On\sphinxquotedblright{} mode and reach our goal in bounded time \(4s\), and set the minimal staying time to \(1s\).

the input file can be written as:

\begin{sphinxVerbatim}[commandchars=\\\{\}]
\PYG{p}{\PYGZob{}}
  \PYG{l+s+s2}{\PYGZdq{}}\PYG{l+s+s2}{modes}\PYG{l+s+s2}{\PYGZdq{}}\PYG{p}{:}\PYG{p}{[}\PYG{l+s+s2}{\PYGZdq{}}\PYG{l+s+s2}{On}\PYG{l+s+s2}{\PYGZdq{}}\PYG{p}{,} \PYG{l+s+s2}{\PYGZdq{}}\PYG{l+s+s2}{Off}\PYG{l+s+s2}{\PYGZdq{}}\PYG{p}{]}\PYG{p}{,}
  \PYG{l+s+s2}{\PYGZdq{}}\PYG{l+s+s2}{initialMode}\PYG{l+s+s2}{\PYGZdq{}}\PYG{p}{:}\PYG{l+s+s2}{\PYGZdq{}}\PYG{l+s+s2}{On}\PYG{l+s+s2}{\PYGZdq{}}\PYG{p}{,}
  \PYG{l+s+s2}{\PYGZdq{}}\PYG{l+s+s2}{variables}\PYG{l+s+s2}{\PYGZdq{}}\PYG{p}{:}\PYG{p}{[}\PYG{l+s+s2}{\PYGZdq{}}\PYG{l+s+s2}{temp}\PYG{l+s+s2}{\PYGZdq{}}\PYG{p}{]}\PYG{p}{,}
  \PYG{l+s+s2}{\PYGZdq{}}\PYG{l+s+s2}{initialSet}\PYG{l+s+s2}{\PYGZdq{}}\PYG{p}{:}\PYG{p}{[}\PYG{p}{[}\PYG{l+m+mf}{75.0}\PYG{p}{]}\PYG{p}{,}\PYG{p}{[}\PYG{l+m+mf}{76.0}\PYG{p}{]}\PYG{p}{]}\PYG{p}{,}
  \PYG{l+s+s2}{\PYGZdq{}}\PYG{l+s+s2}{unsafeSet}\PYG{l+s+s2}{\PYGZdq{}}\PYG{p}{:}\PYG{l+s+s2}{\PYGZdq{}}\PYG{l+s+s2}{@Allmode:temp\PYGZgt{}90}\PYG{l+s+s2}{\PYGZdq{}}\PYG{p}{,}
  \PYG{l+s+s2}{\PYGZdq{}}\PYG{l+s+s2}{goalSet}\PYG{l+s+s2}{\PYGZdq{}}\PYG{p}{:}\PYG{l+s+s2}{\PYGZdq{}}\PYG{l+s+s2}{And(temp\PYGZgt{}=68.0, temp\PYGZlt{}=72.0)}\PYG{l+s+s2}{\PYGZdq{}}\PYG{p}{,}
  \PYG{l+s+s2}{\PYGZdq{}}\PYG{l+s+s2}{timeHorizon}\PYG{l+s+s2}{\PYGZdq{}}\PYG{p}{:}\PYG{l+m+mf}{4.0}\PYG{p}{,}
  \PYG{l+s+s2}{\PYGZdq{}}\PYG{l+s+s2}{minTimeThres}\PYG{l+s+s2}{\PYGZdq{}}\PYG{p}{:}\PYG{l+m+mf}{1.0}\PYG{p}{,}
  \PYG{l+s+s2}{\PYGZdq{}}\PYG{l+s+s2}{directory}\PYG{l+s+s2}{\PYGZdq{}}\PYG{p}{:}\PYG{l+s+s2}{\PYGZdq{}}\PYG{l+s+s2}{examples/Thermostats}\PYG{l+s+s2}{\PYGZdq{}}\PYG{p}{,}
  \PYG{l+s+s2}{\PYGZdq{}}\PYG{l+s+s2}{goal}\PYG{l+s+s2}{\PYGZdq{}}\PYG{p}{:}\PYG{p}{[}\PYG{p}{[}\PYG{l+s+s2}{\PYGZdq{}}\PYG{l+s+s2}{temp}\PYG{l+s+s2}{\PYGZdq{}}\PYG{p}{]}\PYG{p}{,}\PYG{p}{[}\PYG{l+m+mf}{68.0}\PYG{p}{]}\PYG{p}{,}\PYG{p}{[}\PYG{l+m+mf}{72.0}\PYG{p}{]}\PYG{p}{]}
\PYG{p}{\PYGZcb{}}
\end{sphinxVerbatim}

Save the input file in the folder input/rrtinput and name it as temp.json.

Run the graph search algorithm using the command:

\begin{sphinxVerbatim}[commandchars=\\\{\}]
\PYG{n}{python} \PYG{n}{rrt}\PYG{o}{.}\PYG{n}{py} \PYG{n+nb}{input}\PYG{o}{/}\PYG{n}{rrtinput}\PYG{o}{/}\PYG{n}{temp}\PYG{o}{.}\PYG{n}{json}
\end{sphinxVerbatim}

The graph has been found with the output:

\begin{sphinxVerbatim}[commandchars=\\\{\}]
goal reached!
\end{sphinxVerbatim}

If you check the the output/rrtGraph.png, you would get a transition graph for this problem. As you can see the system turn from On state to Off state to reach the goal.
\begin{figure}[htbp]
\centering
\capstart

\noindent\sphinxincludegraphics[scale=0.6]{{rrtGraph}.png}
\caption{The white box transition graph of the thermostat system}\label{\detokenize{dryvr's_control_synthesis:id1}}\end{figure}


\chapter{Examples}
\label{\detokenize{example::doc}}\label{\detokenize{example:examples}}\label{\detokenize{example:example-label}}

\section{Getting started: Simple Automatic Emergency Braking}
\label{\detokenize{example:getting-started-simple-automatic-emergency-braking}}\begin{figure}[htbp]
\centering
\capstart

\noindent\sphinxincludegraphics[scale=0.3]{{Two_cars}.png}
\caption{An illustration of Automatic Emergency Braking System}\label{\detokenize{example:id1}}\end{figure}

Consider the example an AEB as shown above:
Cars 1 and 2 are cruising down the highway with zero relative velocity and certain initial relative separation;  Car 1 suddenly switches to a braking mode and starts slowing down according, certain amount of time elapses,  before Car 2 switches to a braking mode. We are interested to analyze the severity (relative velocity) of any possible collisions.


\subsection{Safety Verification of the AEB System}
\label{\detokenize{example:safety-verification-of-the-aeb-system}}
The black-box of the vehicle dynamics is described in \DUrole{xref,std,std-ref}{ADAS-label}, and the transition graph of the above AEB is shown in {\hyperref[\detokenize{dryvr's_language:transition-graph-label}]{\sphinxcrossref{\DUrole{std,std-ref}{Transition Graph}}}}. The unsafe region is that the relative distance between the two cars are too close (\(|sy_1-sy_2|<3\)). The input files describing the hybrid system is shown in {\hyperref[\detokenize{dryvr's_language:input-format-label}]{\sphinxcrossref{\DUrole{std,std-ref}{Input Format}}}}.


\subsection{Verification Result of the AEB System}
\label{\detokenize{example:verification-result-of-the-aeb-system}}
Run DryVR's verification algorithm for the AEB system:

\begin{sphinxVerbatim}[commandchars=\\\{\}]
\PYG{n}{python} \PYG{n}{main}\PYG{o}{.}\PYG{n}{py} \PYG{n+nb}{input}\PYG{o}{/}\PYG{n}{daginput}\PYG{o}{/}\PYG{n}{input\PYGZus{}brake}\PYG{o}{.}\PYG{n}{json}
\end{sphinxVerbatim}

The system is checked to be safe. We can also plot the reachtubes for different variables. For example, the reachtubes for the position of Car1 and Car2 along the road the direction are shown below. From the reachtube we can also clearly see that the relative distance between the two cars are never too small.
\begin{figure}[htbp]
\centering
\capstart

\noindent\sphinxincludegraphics{{v2}.png}
\caption{Reachtube of the position sy of Car1 and Car2}\label{\detokenize{example:id2}}\end{figure}


\section{Verification Peformance}
\label{\detokenize{example:verification-peformance}}
We have measured performance for examples come with DryVR 2.0.
Peformance is measured using computer with i7 6600u, 16gb ram, Ubuntu 16.04 OS.

\noindent\begin{tabulary}{\linewidth}{|L|L|L|L|L|L|}
\hline

Model
&
Dimension
&
Simulation time
&
Verfication Time
&
Total Time
&
Flow* time
\\
\hline
Biological model I
&
7
&
0.01s
&
0.03s
&
0.04s
&
66.4s
\\
\hline
Biological model II
&
7
&
0.01s
&
0.03s
&
0.04s
&
223.4s
\\
\hline
Coupled Vanderpol
&
4
&
0.03s
&
0.11s
&
0.14s
&
1038.3s
\\
\hline
Spring pendulum
&
4
&
0.05s
&
0.11s
&
0.16s
&
1377.5s
\\
\hline
Roessler
&
3
&
0.02s
&
0.34s
&
0.36s
&
17.1s
\\
\hline
Lorentz system
&
3
&
0.34s
&
0.73s
&
1.07s
&
316.7s
\\
\hline
Lac operon
&
2
&
0.47s
&
170.88s
&
171.35s
&
44.2s
\\
\hline
Lotka-Volterra
&
2
&
0.02s
&
0.08s
&
0.10s
&
3.9s
\\
\hline
Buckling column
&
2
&
0.04s
&
0.39s
&
0.43s
&
26.4s
\\
\hline
Jet engine
&
2
&
0.07s
&
12.03s
&
12.1s
&
6.8s
\\
\hline
Brusselator
&
2
&
0.10s
&
2.92s
&
3.02s
&
5.2s
\\
\hline
Vanderpol
&
2
&
0.05s
&
2.87s
&
2.92s
&
6.4s
\\
\hline
Vehicle platoon 3
&
9
&
0.32s
&
3.96s
&
4.28s
&
21.08s
\\
\hline
Uniform nor sigmoid
&
3
&
120.91s
&
1193.31s
&
1314.22s
&
Exception
\\
\hline
Uniform inverter loop
&
2
&
10.94s
&
267.62
&
278.56s
&
Exception
\\
\hline
Uniform inverter sigmoid
&
2
&
24.87s
&
221.94s
&
246.76s
&
Exception
\\
\hline
Uniform nor ramp
&
3
&
173.77s
&
1591.78s
&
1765.55s
&
Exception
\\
\hline
Uniform or ramp
&
4
&
176.70s
&
1602.17s
&
1778.87s
&
Exception
\\
\hline
Uniform or sigmoid
&
4
&
168.75s
&
2017.25s
&
2186.00s
&
Exception
\\
\hline
Clamped beam
&
348
&
540.80s
&
5176.83s
&
5717.63s
&
Time out
\\
\hline
Building model
&
48
&
3.28s
&
16.96s
&
20.24s
&
Time out
\\
\hline
Partial differential equation
&
20
&
12.05s
&
29.16s
&
41.21s
&
Time out
\\
\hline
FOM
&
20
&
12.18s
&
28.79s
&
40.9s
&
Time out
\\
\hline
Motor control system
&
8
&
5.22s
&
12.67s
&
17.89s
&
Time out
\\
\hline
International space station
&
25
&
79.99s
&
193.61s
&
243.60s
&
Time out
\\
\hline
Lane merge
&
8
&
0.29s
&
563.23s
&
563.52s
&
N/A
\\
\hline\end{tabulary}



\section{Graph Search Performance}
\label{\detokenize{example:graph-search-performance}}
Peformance is measured using computer with i7 6600u, 16gb ram, Ubuntu 16.04 OS.
Note the running time for graph search can be very different since the alogirthm is randomly search for the graph. It may also return nothing as well. Try to run algorithm multiple times if it does not return the graph.

\noindent\begin{tabulary}{\linewidth}{|L|L|L|L|L|}
\hline

Example
&
Dimension
&
Time horizon
&
Min staying time
&
Running Time
\\
\hline
vehicle collision avoidance
&
4
&
50.0s
&
2.0s
&
1896.26s
\\
\hline
robot in maze
&
4
&
10.0s
&
1.0s
&
98.93s
\\
\hline
motion plan
&
3
&
6.0s
&
1.0s
&
4.55s
\\
\hline
DC motor
&
2
&
1.0s
&
0.1s
&
0.35s
\\
\hline
room heating
&
3
&
25.0s
&
2.0s
&
2.66s
\\
\hline
inverted pendulum
&
2
&
2.0s
&
0.2s
&
6.06s
\\
\hline\end{tabulary}



\chapter{Publications}
\label{\detokenize{publications::doc}}\label{\detokenize{publications:publications}}\begin{itemize}
\item {} 
Chuchu Fan, Bolun Qi, Sayan Mitra and Mahesh Viswanathan, \sphinxhref{https://link.springer.com/chapter/10.1007\%2F978-3-319-63387-9\_22}{DRYVR:Data-driven verification and compositional reasoning for automotive systems}, CAV 2017. {[}\sphinxhref{https://www.youtube.com/watch?v=9j7KcbZx6m0}{Video}{]}

\item {} 
Chuchu Fan, Bolun Qi and Sayan Mitra, \sphinxhref{https://arxiv.org/abs/1704.06406}{Road to safe autonomy with data and formal reasoning}, (To appear in IEEE Design \& Test).

\end{itemize}


\chapter{People Involved}
\label{\detokenize{contact:people-involved}}\label{\detokenize{contact::doc}}
If you have any problem using the DryVR, contact the authors of the accompanying paper(s)

\sphinxhref{http://cfan10.web.engr.illinois.edu/}{Chuchu Fan}
PhD candidate, ECE, \sphinxhref{mailto:cfan10@illinois.edu}{Email}

\sphinxhref{https://www.linkedin.com/in/bolun-qi-28483bb9/}{Bolun Qi}
Graduate student, ECE, \sphinxhref{mailto:bolunqi2@illinois.edu}{Email}

\sphinxhref{http://mitras.ece.illinois.edu/}{Sayan Mitra}
Associate Professor, ECE, \sphinxhref{mailto:mitras@illinois.edu}{Email}

\sphinxhref{http://vmahesh.cs.illinois.edu/}{Mahesh Viswanathan}
Professor, CS, \sphinxhref{mailto:vmahesh@illinois.edu}{Email}



\renewcommand{\indexname}{Index}
\printindex
\end{document}